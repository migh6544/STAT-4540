% Options for packages loaded elsewhere
\PassOptionsToPackage{unicode}{hyperref}
\PassOptionsToPackage{hyphens}{url}
%
\documentclass[
]{article}
\usepackage{amsmath,amssymb}
\usepackage{lmodern}
\usepackage{ifxetex,ifluatex}
\ifnum 0\ifxetex 1\fi\ifluatex 1\fi=0 % if pdftex
  \usepackage[T1]{fontenc}
  \usepackage[utf8]{inputenc}
  \usepackage{textcomp} % provide euro and other symbols
\else % if luatex or xetex
  \usepackage{unicode-math}
  \defaultfontfeatures{Scale=MatchLowercase}
  \defaultfontfeatures[\rmfamily]{Ligatures=TeX,Scale=1}
\fi
% Use upquote if available, for straight quotes in verbatim environments
\IfFileExists{upquote.sty}{\usepackage{upquote}}{}
\IfFileExists{microtype.sty}{% use microtype if available
  \usepackage[]{microtype}
  \UseMicrotypeSet[protrusion]{basicmath} % disable protrusion for tt fonts
}{}
\makeatletter
\@ifundefined{KOMAClassName}{% if non-KOMA class
  \IfFileExists{parskip.sty}{%
    \usepackage{parskip}
  }{% else
    \setlength{\parindent}{0pt}
    \setlength{\parskip}{6pt plus 2pt minus 1pt}}
}{% if KOMA class
  \KOMAoptions{parskip=half}}
\makeatother
\usepackage{xcolor}
\IfFileExists{xurl.sty}{\usepackage{xurl}}{} % add URL line breaks if available
\IfFileExists{bookmark.sty}{\usepackage{bookmark}}{\usepackage{hyperref}}
\hypersetup{
  pdftitle={{[}STAT 4540{]} HW-2},
  pdfauthor={Michael Ghattas},
  hidelinks,
  pdfcreator={LaTeX via pandoc}}
\urlstyle{same} % disable monospaced font for URLs
\usepackage[margin=1in]{geometry}
\usepackage{color}
\usepackage{fancyvrb}
\newcommand{\VerbBar}{|}
\newcommand{\VERB}{\Verb[commandchars=\\\{\}]}
\DefineVerbatimEnvironment{Highlighting}{Verbatim}{commandchars=\\\{\}}
% Add ',fontsize=\small' for more characters per line
\usepackage{framed}
\definecolor{shadecolor}{RGB}{248,248,248}
\newenvironment{Shaded}{\begin{snugshade}}{\end{snugshade}}
\newcommand{\AlertTok}[1]{\textcolor[rgb]{0.94,0.16,0.16}{#1}}
\newcommand{\AnnotationTok}[1]{\textcolor[rgb]{0.56,0.35,0.01}{\textbf{\textit{#1}}}}
\newcommand{\AttributeTok}[1]{\textcolor[rgb]{0.77,0.63,0.00}{#1}}
\newcommand{\BaseNTok}[1]{\textcolor[rgb]{0.00,0.00,0.81}{#1}}
\newcommand{\BuiltInTok}[1]{#1}
\newcommand{\CharTok}[1]{\textcolor[rgb]{0.31,0.60,0.02}{#1}}
\newcommand{\CommentTok}[1]{\textcolor[rgb]{0.56,0.35,0.01}{\textit{#1}}}
\newcommand{\CommentVarTok}[1]{\textcolor[rgb]{0.56,0.35,0.01}{\textbf{\textit{#1}}}}
\newcommand{\ConstantTok}[1]{\textcolor[rgb]{0.00,0.00,0.00}{#1}}
\newcommand{\ControlFlowTok}[1]{\textcolor[rgb]{0.13,0.29,0.53}{\textbf{#1}}}
\newcommand{\DataTypeTok}[1]{\textcolor[rgb]{0.13,0.29,0.53}{#1}}
\newcommand{\DecValTok}[1]{\textcolor[rgb]{0.00,0.00,0.81}{#1}}
\newcommand{\DocumentationTok}[1]{\textcolor[rgb]{0.56,0.35,0.01}{\textbf{\textit{#1}}}}
\newcommand{\ErrorTok}[1]{\textcolor[rgb]{0.64,0.00,0.00}{\textbf{#1}}}
\newcommand{\ExtensionTok}[1]{#1}
\newcommand{\FloatTok}[1]{\textcolor[rgb]{0.00,0.00,0.81}{#1}}
\newcommand{\FunctionTok}[1]{\textcolor[rgb]{0.00,0.00,0.00}{#1}}
\newcommand{\ImportTok}[1]{#1}
\newcommand{\InformationTok}[1]{\textcolor[rgb]{0.56,0.35,0.01}{\textbf{\textit{#1}}}}
\newcommand{\KeywordTok}[1]{\textcolor[rgb]{0.13,0.29,0.53}{\textbf{#1}}}
\newcommand{\NormalTok}[1]{#1}
\newcommand{\OperatorTok}[1]{\textcolor[rgb]{0.81,0.36,0.00}{\textbf{#1}}}
\newcommand{\OtherTok}[1]{\textcolor[rgb]{0.56,0.35,0.01}{#1}}
\newcommand{\PreprocessorTok}[1]{\textcolor[rgb]{0.56,0.35,0.01}{\textit{#1}}}
\newcommand{\RegionMarkerTok}[1]{#1}
\newcommand{\SpecialCharTok}[1]{\textcolor[rgb]{0.00,0.00,0.00}{#1}}
\newcommand{\SpecialStringTok}[1]{\textcolor[rgb]{0.31,0.60,0.02}{#1}}
\newcommand{\StringTok}[1]{\textcolor[rgb]{0.31,0.60,0.02}{#1}}
\newcommand{\VariableTok}[1]{\textcolor[rgb]{0.00,0.00,0.00}{#1}}
\newcommand{\VerbatimStringTok}[1]{\textcolor[rgb]{0.31,0.60,0.02}{#1}}
\newcommand{\WarningTok}[1]{\textcolor[rgb]{0.56,0.35,0.01}{\textbf{\textit{#1}}}}
\usepackage{graphicx}
\makeatletter
\def\maxwidth{\ifdim\Gin@nat@width>\linewidth\linewidth\else\Gin@nat@width\fi}
\def\maxheight{\ifdim\Gin@nat@height>\textheight\textheight\else\Gin@nat@height\fi}
\makeatother
% Scale images if necessary, so that they will not overflow the page
% margins by default, and it is still possible to overwrite the defaults
% using explicit options in \includegraphics[width, height, ...]{}
\setkeys{Gin}{width=\maxwidth,height=\maxheight,keepaspectratio}
% Set default figure placement to htbp
\makeatletter
\def\fps@figure{htbp}
\makeatother
\setlength{\emergencystretch}{3em} % prevent overfull lines
\providecommand{\tightlist}{%
  \setlength{\itemsep}{0pt}\setlength{\parskip}{0pt}}
\setcounter{secnumdepth}{-\maxdimen} % remove section numbering
\ifluatex
  \usepackage{selnolig}  % disable illegal ligatures
\fi

\title{{[}STAT 4540{]} HW-2}
\author{Michael Ghattas}
\date{2/9/2022}

\begin{document}
\maketitle

\hypertarget{problem-1}{%
\section{Problem 1}\label{problem-1}}

\(X_{t} + \frac{5}{4} X_{t - 1} = \frac{1}{2} \tilde{Z}_{t} + \frac{3}{4} \tilde{Z}_{t - 1}\)\\

let \(Z_{t} = \frac{1}{2} \tilde{Z}_{t}\)\\

\(\tilde{Z}_{t} \sim wn(0, \sigma^2) \to Z_{t} \sim wn(0, \frac{1}{4} \sigma^2)\)\\

\(\to \frac{1}{2} \tilde{Z}_{t} + \frac{3}{4} \tilde{Z}_{t - 1} = Z_{t} + \frac{\frac{3}{4}}{2} Z_{t - 1} = Z_{t} + \frac{3}{8} Z_{t - 1}\)\\

\(\to X_{t} + \frac{5}{4} X_{t - 1} = Z_{t} + \frac{3}{8} Z_{t - 1}\)\\

Let \(\frac{5}{4} X_{t - 1} = -(-\frac{5}{4}) X_{t - 1}\)\\

\(\to X_{t} - (-\frac{5}{4}) X_{t - 1} = Z_{t} + \frac{3}{8} Z_{t - 1}\)\\

\hypertarget{a}{%
\subsubsection{(a)}\label{a}}

Given the below points, we can conclude that there exists a unique and
stationary solution for \(\{X_{t}\}\):\\

\begin{enumerate}
\def\labelenumi{\arabic{enumi}.}
\item
  \(X_{t} - (-\frac{5}{4}) X_{t - 1} = Z_{t} + \frac{3}{8} Z_{t - 1}\)\\
\item
  \(\phi = -\frac{5}{4}\) and
  \(\theta = \frac{3}{8} \to \phi + \theta = \frac{3}{8} - \frac{5}{4} = \frac{3 - 10}{8} = -\frac{7}{8} \neq 0\)\\
\item
  \(\phi = -\frac{5}{4} \to |\phi| \neq 1\)\\
\end{enumerate}

\hypertarget{b}{%
\subsubsection{(b)}\label{b}}

Given that \(\phi = -\frac{5}{4} \to |\phi| > 1\), thus the process is
non-causal.\\

\hypertarget{c}{%
\subsubsection{(c)}\label{c}}

Given that \(\theta = \frac{3}{8} \to |\theta| < 1\), thus the process
is invertible.\\

\hypertarget{d}{%
\subsubsection{(d)}\label{d}}

\(X_{t} - (-\frac{5}{4}) X_{t - 1} = Z_{t} + \frac{3}{8} Z_{t - 1}\)\\

\begin{align*}
X_{t} &= - (\frac{\frac{3}{8}}{-\frac{5}{4}}) Z_{t} - (\frac{3}{8} + (- \frac{5}{4})) \sum_{j = 0}^{1} \frac{1}{\phi^{j + 1}} Z_{t + j} \\
&= - (- \frac{40}{12}) Z_{t} - (\frac{3}{8} - \frac{5}{4}) \sum_{j = 0}^{1} \frac{1}{(\frac{5}{4})^{j + 1}} Z_{t + j} \\
&= \frac{40}{12} Z_{t} - (\frac{3 - 10}{8}) \cdot (\frac{Z_{t + 0}}{(\frac{5}{4})^{0 + 1}} + \frac{Z_{t + 1}}{(\frac{5}{4})^{1 + 1}}) \\
&= \frac{10}{3} Z_{t} - (-\frac{7}{8}) \cdot (\frac{Z_{t}}{(\frac{5}{4})^{1}} + \frac{Z_{t + 1}}{(\frac{5}{4})^{2}}) \\
&= \frac{10}{3} Z_{t} + \frac{7}{8}  \cdot (\frac{Z_{t}}{\frac{5}{4}} + \frac{Z_{t + 1}}{\frac{25}{26}}) \\
&= \frac{10}{3} Z_{t} + \frac{7}{8}  \cdot (\frac{4}{5} Z_{t} + \frac{25}{16} Z_{t + 1}) \\
&= \frac{10}{3} Z_{t} + \frac{28}{40} Z_{t} + \frac{175}{128} Z_{t + 1} \\
&= (\frac{100 + 21}{30}) Z_{t} + \frac{175}{128} Z_{t + 1} \\
&= \frac{121}{30} Z_{t} + \frac{175}{128} Z_{t + 1} \\
&= \frac{121}{30} \cdot (\frac{1}{2} \tilde{Z}_{t}) + \frac{175}{128} \cdot (\frac{1}{2} \tilde{Z}_{t + 1}) \\
&= \frac{121}{60} \tilde{Z}_{t} + \frac{175}{256} \tilde{Z}_{t + 1} \\
\end{align*}

Thus:
\(X_{t} = \frac{121}{60} \tilde{Z}_{t} + \frac{175}{256} \tilde{Z}_{t + 1}\).\\

\hypertarget{problem-2}{%
\section{Problem 2}\label{problem-2}}

\hypertarget{a-1}{%
\subsubsection{(a)}\label{a-1}}

The big picture is filtering the estimated trend from the data through
filtering out until we are left with nothing but residuals that are
stationary.\\

\hypertarget{b-1}{%
\subsubsection{(b)}\label{b-1}}

No, the shift by re-indexing should have not affect any significant
change.\\

\hypertarget{c-1}{%
\subsubsection{(c)}\label{c-1}}

The condition should not depend on the indexing choice, however the
output would change if we redefined the indexing sequence.\\

\hypertarget{problem-3}{%
\section{Problem 3}\label{problem-3}}

\hypertarget{a-2}{%
\subsubsection{(a)}\label{a-2}}

\begin{Shaded}
\begin{Highlighting}[]
\FunctionTok{library}\NormalTok{(lubridate)}
\end{Highlighting}
\end{Shaded}

\begin{verbatim}
## 
## Attaching package: 'lubridate'
\end{verbatim}

\begin{verbatim}
## The following objects are masked from 'package:base':
## 
##     date, intersect, setdiff, union
\end{verbatim}

\begin{Shaded}
\begin{Highlighting}[]
\FunctionTok{library}\NormalTok{(gridExtra)}
\FunctionTok{library}\NormalTok{(ggplot2)}
\FunctionTok{library}\NormalTok{(dplyr)}
\end{Highlighting}
\end{Shaded}

\begin{verbatim}
## Warning: package 'dplyr' was built under R version 4.1.2
\end{verbatim}

\begin{verbatim}
## 
## Attaching package: 'dplyr'
\end{verbatim}

\begin{verbatim}
## The following object is masked from 'package:gridExtra':
## 
##     combine
\end{verbatim}

\begin{verbatim}
## The following objects are masked from 'package:stats':
## 
##     filter, lag
\end{verbatim}

\begin{verbatim}
## The following objects are masked from 'package:base':
## 
##     intersect, setdiff, setequal, union
\end{verbatim}

\begin{Shaded}
\begin{Highlighting}[]
\NormalTok{data }\OtherTok{\textless{}{-}} \FunctionTok{load}\NormalTok{(}\StringTok{"/Users/Home/Documents/Michael\_Ghattas/School/CU\_Boulder/2022/Spring 2022/STAT {-} 4540/HW/2/DailyMaxMelbourne19851987.RData"}\NormalTok{)}
\NormalTok{dates }\OtherTok{=} \FunctionTok{as.Date}\NormalTok{(dates,}\AttributeTok{format =} \StringTok{"\%m/\%d/\%y"}\NormalTok{)}
\NormalTok{df }\OtherTok{=} \FunctionTok{data.frame}\NormalTok{(dates, temp)}
\FunctionTok{plot}\NormalTok{(dates, temp, }\AttributeTok{xlab =} \StringTok{"Date"}\NormalTok{, }\AttributeTok{ylab =} \StringTok{"Degrees Celsius"}\NormalTok{, }\AttributeTok{main =} \StringTok{"Daily max temperature in Melbourne"}\NormalTok{, }\AttributeTok{pch =} \DecValTok{01}\NormalTok{, }\AttributeTok{cex =} \FloatTok{0.5}\NormalTok{)}
\end{Highlighting}
\end{Shaded}

\includegraphics{{[}STAT-4540{]}-HW-2---Michael-Ghattas_files/figure-latex/unnamed-chunk-1-1.pdf}

\hypertarget{b-2}{%
\subsubsection{(b)}\label{b-2}}

\begin{Shaded}
\begin{Highlighting}[]
\NormalTok{df.ts }\OtherTok{\textless{}{-}} \FunctionTok{ts}\NormalTok{(}\AttributeTok{data =}\NormalTok{ df}\SpecialCharTok{$}\NormalTok{temp, }\AttributeTok{start =} \FunctionTok{c}\NormalTok{(}\DecValTok{1985}\NormalTok{, }\DecValTok{1}\NormalTok{), }\AttributeTok{end =} \FunctionTok{c}\NormalTok{(}\DecValTok{1987}\NormalTok{, }\DecValTok{365}\NormalTok{), }\AttributeTok{frequency =} \DecValTok{365}\NormalTok{)}
\NormalTok{out }\OtherTok{\textless{}{-}} \FunctionTok{decompose}\NormalTok{(df.ts, }\AttributeTok{type =} \StringTok{"additive"}\NormalTok{)}
\FunctionTok{plot}\NormalTok{(out)}
\end{Highlighting}
\end{Shaded}

\includegraphics{{[}STAT-4540{]}-HW-2---Michael-Ghattas_files/figure-latex/unnamed-chunk-2-1.pdf}

We can see that the decomposition did not work well, as the results are
to noisy. An appropriate filter is needed prior to the decompose
function.\\

\hypertarget{c-2}{%
\subsubsection{(c)}\label{c-2}}

\begin{Shaded}
\begin{Highlighting}[]
\FunctionTok{acf}\NormalTok{(out}\SpecialCharTok{$}\NormalTok{random, }\AttributeTok{na.action =}\NormalTok{ na.pass)}
\end{Highlighting}
\end{Shaded}

\includegraphics{{[}STAT-4540{]}-HW-2---Michael-Ghattas_files/figure-latex/unnamed-chunk-3-1.pdf}

\begin{Shaded}
\begin{Highlighting}[]
\FunctionTok{acf}\NormalTok{(out}\SpecialCharTok{$}\NormalTok{random, }\AttributeTok{na.action =}\NormalTok{ na.pass, }\AttributeTok{lag.max =} \DecValTok{500}\NormalTok{)}
\end{Highlighting}
\end{Shaded}

\includegraphics{{[}STAT-4540{]}-HW-2---Michael-Ghattas_files/figure-latex/unnamed-chunk-3-2.pdf}

The results present moderate correlation, thus further filtering is
needed to be able to identify white noise. A low-degree polynomial based
regression model might be helpful.\\

\hypertarget{d-1}{%
\subsubsection{(d)}\label{d-1}}

\begin{Shaded}
\begin{Highlighting}[]
\NormalTok{df}\SpecialCharTok{$}\NormalTok{year }\OtherTok{\textless{}{-}} \FunctionTok{year}\NormalTok{(df}\SpecialCharTok{$}\NormalTok{dates)}
\NormalTok{df}\SpecialCharTok{$}\NormalTok{month }\OtherTok{\textless{}{-}} \FunctionTok{month}\NormalTok{(df}\SpecialCharTok{$}\NormalTok{dates)}

\NormalTok{avgtemp }\OtherTok{\textless{}{-}} \FunctionTok{aggregate}\NormalTok{(temp }\SpecialCharTok{\textasciitilde{}}\NormalTok{ year }\SpecialCharTok{+}\NormalTok{ month, df, mean)}
\NormalTok{avgtemp}\SpecialCharTok{$}\NormalTok{date }\OtherTok{=} \FunctionTok{as.Date}\NormalTok{(}\FunctionTok{paste}\NormalTok{(avgtemp}\SpecialCharTok{$}\NormalTok{year, avgtemp}\SpecialCharTok{$}\NormalTok{month, }\DecValTok{01}\NormalTok{), }\StringTok{"\%y \%m \%d"}\NormalTok{)}
\NormalTok{avgtemp }\OtherTok{=} \FunctionTok{arrange}\NormalTok{(avgtemp, date)}

\NormalTok{avgtemp.ts }\OtherTok{\textless{}{-}} \FunctionTok{ts}\NormalTok{(}\AttributeTok{data =}\NormalTok{ avgtemp}\SpecialCharTok{$}\NormalTok{temp, }\AttributeTok{start =} \FunctionTok{c}\NormalTok{(}\DecValTok{1985}\NormalTok{, }\DecValTok{1}\NormalTok{), }\AttributeTok{end =} \FunctionTok{c}\NormalTok{(}\DecValTok{1987}\NormalTok{, }\DecValTok{12}\NormalTok{), }\AttributeTok{frequency =} \DecValTok{12}\NormalTok{)}
\NormalTok{avgtemp.out }\OtherTok{\textless{}{-}} \FunctionTok{decompose}\NormalTok{(df.ts, }\AttributeTok{type =} \StringTok{"additive"}\NormalTok{)}

\FunctionTok{plot}\NormalTok{(df}\SpecialCharTok{$}\NormalTok{month, temp, }\AttributeTok{xlab =} \StringTok{"Month"}\NormalTok{, }\AttributeTok{ylab =} \StringTok{"Degrees Celsius"}\NormalTok{, }\AttributeTok{main =} \StringTok{"Average Monthly Temp"}\NormalTok{, }\AttributeTok{pch =} \DecValTok{01}\NormalTok{, }\AttributeTok{cex =} \FloatTok{0.5}\NormalTok{)}
\end{Highlighting}
\end{Shaded}

\includegraphics{{[}STAT-4540{]}-HW-2---Michael-Ghattas_files/figure-latex/unnamed-chunk-4-1.pdf}

\begin{Shaded}
\begin{Highlighting}[]
\FunctionTok{plot}\NormalTok{(avgtemp.out)}
\end{Highlighting}
\end{Shaded}

\includegraphics{{[}STAT-4540{]}-HW-2---Michael-Ghattas_files/figure-latex/unnamed-chunk-4-2.pdf}

\begin{Shaded}
\begin{Highlighting}[]
\FunctionTok{acf}\NormalTok{(avgtemp.out}\SpecialCharTok{$}\NormalTok{random, }\AttributeTok{na.action =}\NormalTok{ na.pass)}
\end{Highlighting}
\end{Shaded}

\includegraphics{{[}STAT-4540{]}-HW-2---Michael-Ghattas_files/figure-latex/unnamed-chunk-4-3.pdf}

From the data we can see a correlation between the month and average
temperature. We can also see a relationship between the the average
temperature of each month in relation to the previous month. Additional
filtering is needed to extract trends, seasonality and noise.\\

\hypertarget{problem-4}{%
\section{Problem 4}\label{problem-4}}

\hypertarget{a-3}{%
\subsubsection{(a)}\label{a-3}}

\begin{align*}
Z_{t} &= (1 - B)(1 - B^{12})Y_{t} \\
&= (1 - B - B^{12} + B^{13})Y_{t} \\
&= Y_{t} - Y_{t - 1} - Y_{t - 12} + Y_{t - 13} \\
&= (Y_{t} - Y_{t - 12}) - (Y_{t - 1} - Y_{t - 13}) \\
&= (a + bt + s_{t} + X_{t} - a - b(t - 12) - s_{t - 12} - X_{t - 12}) - (a + b(t - 1) + s_{t - 1} + X_{t - 1} - a - b(t - 13) - s_{t - 13} - X_{t - 13}) \\
&= (bt + X_{t} - b(t - 12) - X_{t - 12}) - (b(t - 1) + X_{t - 1} - b(t - 13) - X_{t - 13}) \\
&= (X_{t} - X_{t - 12} + 12b) - (X_{t - 1} - X_{t - 13} + 13b) \\
&= X_{t} - X_{t - 12} + 12b - X_{t - 1} + X_{t - 13} - 13b) \\
&= X_{t} - X_{t - 1} - X_{t - 12} + X_{t - 13} - b \\
&= X_{t} - X_{t - 1} - X_{t - 12} + X_{t - 13} \\
\end{align*}

Thus: \(Z_{t} = X_{t} - X_{t - 1} - X_{t - 12} + X_{t - 13}\) satisfies
an AR(p) stationary process.\\

\end{document}
